\documentclass[a4paper,11pt]{article}
\usepackage[osf]{mathpazo}
\usepackage{ms}
\usepackage[]{natbib}
\usepackage{booktabs}
\raggedright

\newcommand{\smurl}[1]{{\footnotesize\url{#1}}}
\usepackage{graphicx}

\title{What we don't currently know about global plant functional diversity}
\author{
Will Cornwell$^1$
et al.}

\affiliation{
*final list and order undecided\\
$^1$ University of NSW\\
}
\date{}

\bibliographystyle{mee}

\usepackage[title,titletoc,toc]{appendix}

\mstype{Research Article}
\runninghead{What we don't know about plant diversity}
\keywords{}

\begin{document}
\mstitlepage
\noindent
% \doublespacing
% \linenumbers

\section{Summary}


\section{Introduction}

\section{Methods}

\section{Results}

\subsection{Species that we know a lot about}

Among the currently accepted 350,699 plant species in the world, there are about 1,647 that we know a lot about, spatially, genetically, and functionally--this is 0.47\% of extant diversity.  With more effort on taxonomy and synonymy that number may be a bit higher with existing data, but largely this represent data that is yet to be collected.

\subsection{Species that we know a little about}

\subsection{Species that we know nothing about}


\section{Discussion}




Data interpolation techniques rely principally on phylogenetic relatedness.  
Most, models of the expected similarity based on relatedness rely on an underlying implicit or explicit model of trait evolution.  
Usefully some of the recently proposed methods do allow for characterising uncertainty, this uncertainty in the typical case
does not include model uncertainty.  And as most simple models of trait evolution for plants are far from adequate (Pennell), and heterogeneity in evolutionary 
process among the branches of the tree is 
we have a ways to go before we fully understand uncertainty in trait imputation approaches.   

\section{Tables}



% latex table generated in R 3.2.2 by xtable 1.8-0 package
% Mon Jan 11 00:57:19 2016
\begin{tabular}{rlrrrrl}
  \toprule
 & family & prop.sampled & sr & g & p & db \\ 
  \midrule
1 & Orchidaceae & 0.10 & 27732 & 2639.25 & 0.00 & try \\ 
  2 & Pottiaceae & 0.00 & 3169 & 1501.83 & 0.00 & try \\ 
  3 & Hypnaceae & 0.00 & 2519 & 1189.80 & 0.00 & try \\ 
  4 & Lejeuneaceae & 0.00 & 2269 & 1084.84 & 0.00 & try \\ 
  5 & Bryaceae & 0.00 & 2103 & 1005.20 & 0.00 & try \\ 
  6 & Dicranaceae & 0.00 & 2116 & 944.73 & 0.00 & try \\ 
  7 & Sematophyllaceae & 0.00 & 1628 & 751.98 & 0.00 & try \\ 
  8 & Acanthaceae & 0.06 & 3948 & 737.07 & 0.00 & try \\ 
  9 & Gesneriaceae & 0.06 & 3124 & 612.32 & 0.00 & try \\ 
  10 & Orthotrichaceae & 0.00 & 1272 & 582.60 & 0.00 & try \\ 
  11 & Orchidaceae & 0.20 & 27732 & 1325.67 & 0.00 & genbank \\ 
  12 & Hypnaceae & 0.05 & 2519 & 942.28 & 0.00 & genbank \\ 
  13 & Pottiaceae & 0.09 & 3169 & 768.81 & 0.00 & genbank \\ 
  14 & Bryaceae & 0.06 & 2103 & 694.35 & 0.00 & genbank \\ 
  15 & Myrtaceae & 0.15 & 5970 & 672.99 & 0.00 & genbank \\ 
  16 & Sematophyllaceae & 0.07 & 1628 & 481.28 & 0.00 & genbank \\ 
  17 & Fissidentaceae & 0.02 & 817 & 413.23 & 0.00 & genbank \\ 
  18 & Dicranaceae & 0.11 & 2116 & 373.83 & 0.00 & genbank \\ 
  19 & Orthotrichaceae & 0.09 & 1272 & 295.92 & 0.00 & genbank \\ 
  20 & Acanthaceae & 0.18 & 3948 & 225.24 & 0.00 & genbank \\ 
  21 & Lejeuneaceae & 0.07 & 2269 & 3096.61 & 0.00 & gbif \\ 
  22 & Hypnaceae & 0.17 & 2519 & 2215.00 & 0.00 & gbif \\ 
  23 & Plagiochilaceae & 0.06 & 524 & 750.84 & 0.00 & gbif \\ 
  24 & Jungermanniaceae & 0.14 & 717 & 718.18 & 0.00 & gbif \\ 
  25 & Lepidoziaceae & 0.10 & 580 & 702.06 & 0.00 & gbif \\ 
  26 & Jubulaceae & 0.07 & 454 & 630.30 & 0.00 & gbif \\ 
  27 & Lophocoleaceae & 0.05 & 381 & 576.38 & 0.00 & gbif \\ 
  28 & Aneuraceae & 0.05 & 297 & 442.50 & 0.00 & gbif \\ 
  29 & Leskeaceae & 0.19 & 383 & 305.61 & 0.00 & gbif \\ 
  30 & Radulaceae & 0.08 & 190 & 242.88 & 0.00 & gbif \\ 
   \bottomrule
\end{tabular}


% latex table generated in R 3.2.2 by xtable 1.8-0 package
% Mon Jan 11 00:58:08 2016
\begin{tabular}{rrll}
  \toprule
 & con & family & db \\ 
  \midrule
1 & 1.00 & Orchidaceae & try \\ 
  2 & 1.00 & Gesneriaceae & try \\ 
  3 & 2.00 & Acanthaceae & try \\ 
  4 & 2.00 & Dryopteridaceae & try \\ 
  5 & 3.00 &  & try \\ 
  6 & 4.00 & Asteraceae & try \\ 
  7 & 4.00 & Orchidaceae & try \\ 
  8 & 5.00 & Orchidaceae & try \\ 
  9 & 5.00 & Lamiaceae & try \\ 
  10 & 6.00 & Orchidaceae & try \\ 
  11 & 6.00 & Lamiaceae & try \\ 
  12 & 7.00 & Rubiaceae & try \\ 
  13 & 7.00 & Orchidaceae & try \\ 
  14 & 8.00 & Acanthaceae & try \\ 
  15 & 8.00 & Apocynaceae & try \\ 
  16 & 9.00 &  & try \\ 
  17 & 1.00 & Orchidaceae & genbank \\ 
  18 & 1.00 & Myrtaceae & genbank \\ 
  19 & 2.00 & Myrtaceae & genbank \\ 
  20 & 2.00 & Araceae & genbank \\ 
  21 & 3.00 &  & genbank \\ 
  22 & 4.00 & Myrtaceae & genbank \\ 
  23 & 4.00 & Asteraceae & genbank \\ 
  24 & 5.00 & Asteraceae & genbank \\ 
  25 & 5.00 & Rosaceae & genbank \\ 
  26 & 6.00 & Myrtaceae & genbank \\ 
  27 & 6.00 & Frankeniaceae & genbank \\ 
  28 & 7.00 & Orchidaceae & genbank \\ 
  29 & 7.00 & Phyllanthaceae & genbank \\ 
  30 & 8.00 & Acanthaceae & genbank \\ 
  31 & 8.00 & Rutaceae & genbank \\ 
  32 & 9.00 &  & genbank \\ 
   \bottomrule
\end{tabular}


% latex table generated in R 3.2.2 by xtable 1.8-0 package
% Mon Jan 11 00:59:24 2016
\begin{tabular}{rr}
  \toprule
 & out \\ 
  \midrule
genbank & 97921 \\ 
  try.all.names & 74327 \\ 
  gbif & 217681 \\ 
  zae & 28843 \\ 
  diaz & 2170 \\ 
  genbank-try.all.names & 42832 \\ 
  genbank-gbif & 81988 \\ 
  genbank-zae & 26416 \\ 
  genbank-diaz & 1888 \\ 
  try.all.names-gbif & 65535 \\ 
  try.all.names-zae & 27078 \\ 
  try.all.names-diaz & 2158 \\ 
  gbif-zae & 27368 \\ 
  gbif-diaz & 2027 \\ 
  zae-diaz & 1695 \\ 
  genbank-try.all.names-gbif & 39650 \\ 
  genbank-try.all.names-zae & 24961 \\ 
  genbank-try.all.names-diaz & 1886 \\ 
  genbank-gbif-zae & 25153 \\ 
  genbank-gbif-diaz & 1777 \\ 
  genbank-zae-diaz & 1568 \\ 
  try.all.names-gbif-zae & 25783 \\ 
  try.all.names-gbif-diaz & 2019 \\ 
  try.all.names-zae-diaz & 1694 \\ 
  gbif-zae-diaz & 1647 \\ 
   \bottomrule
\end{tabular}


\section{Figures}

\begin{figure}[h!]
\centering
  \includegraphics[width=\textwidth]{figures/multi_gam.png}
\end{figure}

\begin{figure}[h!]
\centering
  \includegraphics[width=\textwidth]{figures/multi_gam_zae_diaz.png}
\end{figure}


\begin{figure}[h!]
\centering
\includegraphics[width=15cm,height=15cm,keepaspectratio]{figures/sampling_in_try_by_family.pdf}
\caption{Size of the boxes are proportional to the species richness of each family.  Proportion of TPL 1.1 accepted names that were in TRY-db as of 2015.  }
\label{fig:try_treemap}
\end{figure}
\clearpage

\bibliography{refs}

\end{document}

